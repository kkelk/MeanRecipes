\documentclass[a4paper,12pt]{article}

\usepackage{geometry}
\usepackage{parskip}

\title{API Documentation}
\author{}
\date{}

\newcommand{\meth}[1]{\textbf{#1}}
\newcommand{\apiendpoint}[3]{%
	\subsection{\texttt{#2}}%
	\framebox[\textwidth][l]{%
		\meth{#1} \parbox[t]{\textwidth}{%
			\texttt{#2}%
			\par%
			#3
		}%
	}%
}
\newcommand{\param}[1]{\texttt{\emph{#1}}}

\begin{document}

\maketitle

\tableofcontents



\section{Endpoints}

\apiendpoint{POST}{/recipe/search/\param{term}/}{
	Compiles an average recipe for the foodstuff, \param{term}.
}

This method takes an optional \meth{GET} parameter, \param{silliness}. This is a number in the range 0--100 which specifies how silly the generated recipe should be, where 0 is the least silly and 100 is the silliest .

It returns a JSON response describing a recipe, with the following schema:

\begin{verbatim}
{
    "title": "Recipe",

    "ingredients": [
        [300.0, "g", "flour"], 
        ...
    ],

    "method": [
        "Step 1",
        "Step 2",
        ...
    ]
}
\end{verbatim}

The \texttt{ingredients} field is a list of arrays of the form \verb![quantity, unit, name]!. The quantities returned are \emph{per person} --- that is, the recipe is written to serve one person. The \texttt{quantity} field may have the value \texttt{null}, indicating that no particular quantity of the ingredient is specified. In this case, the value of the \texttt{unit} field is undefined; the \texttt{name} field will still contain a valid value.

The \texttt{method} field is an ordered list of steps in the recipe.

\end{document}
